%%%%%%% --------------------------------------------------------------------------------------
%%%%%%%  STARTING HERE, DO NOT TOUCH ANYTHING 
%%%%%%% --------------------------------------------------------------------------------------

%%%% Define Document type
\documentclass[10.5pt,a4]{article}

%%%% Include Packages
\usepackage{latexsym}
\newcommand{\bold}[1]{ {\bfseries #1}}
\usepackage[empty]{fullpage}
\usepackage{titlesec}
 \usepackage{marvosym}
\usepackage[usenames,dvipsnames]{color}
\usepackage{verbatim}
\usepackage[hidelinks]{hyperref}
\usepackage{fancyhdr}
\usepackage{multicol}
\usepackage{hyperref}
\usepackage{csquotes}
\usepackage{tabularx}
\hypersetup{colorlinks=true,urlcolor=black}
\usepackage[11pt]{moresize}
\usepackage{setspace}
% \usepackage{fontspec}
\usepackage[inline]{enumitem}
\usepackage{ragged2e}
\usepackage{anyfontsize}

%%%% Set Margins
\usepackage[margin=2cm]{geometry}

%%%% Set Page Style
\pagestyle{fancy}
\fancyhf{} 
\fancyfoot{}
\renewcommand{\headrulewidth}{0pt}
\renewcommand{\footrulewidth}{0pt}

%%%% Set URL Style
\urlstyle{same}

%%%% Set Indentation
\raggedbottom
\raggedright
\setlength{\tabcolsep}{0in}

%%%% Set Secondary Color, Page Number Color, Footer Text
\definecolor{UI_blue}{RGB}{32, 64, 151}
\definecolor{HF_color}{RGB}{179, 179, 179}


%%%% Set Heading Format
\titleformat{\section}{
\color{UI_blue} \scshape \raggedright \large 
}{}{0em}{}[\vspace{-0.7cm} \hrulefill \vspace{-0.2cm}]
%%%%%%% --------------------------------------------------------------------------------------
%%%%%%% --------------------------------------------------------------------------------------
%%%%%%%  END OF "DO NOT TOUCH" REGION
%%%%%%% --------------------------------------------------------------------------------------
%%%%%%% --------------------------------------------------------------------------------------



\begin{document}
%%%%%%% --------------------------------------------------------------------------------------
%%%%%%%  HEADER
%%%%%%% --------------------------------------------------------------------------------------
\begin{center}
    \begin{minipage}[b]{0.24\textwidth}
            \large (267)-401-2843 \\
            \large \href{mailto:L.tsai@mail.utoronto.ca}{L.tsai@mail.utoronto.ca} 
    \end{minipage}% 
    \begin{minipage}[b]{0.5\textwidth}
            \center{     
            \Huge{   Lawrence Tsai}} \\ %
            \vspace{0.1cm}
            {\color{UI_blue} \Large{}}
    \end{minipage}% 
    \begin{minipage}[b]{0.33\textwidth}
            \flushright \small
            {\href{www.linkedin.com/in/lawtsai/}{www.linkedin.com/in/lawtsai/} } \\
            \href{https://github.com/lt77777}{https://github.com/lt77777} \\
            \href{https://lt77777.github.io/}
            {https://lt77777.github.io/}
    \end{minipage}   
    
\vspace{-0.15cm} 
{\color{UI_blue} \hrulefill}
\end{center}

\justify
\setlength{\parindent}{0pt}
\setlength{\parskip}{12pt}
\vspace{0.2cm}


%%%%%%% --------------------------------------------------------------------------------------
%%%%%%%  First 2 Lines
%%%%%%% --------------------------------------------------------------------------------------


Dear Hiring Manager:

I am thrilled to apply to work for you as a \textbf{New Grad starting in January or February 2024}. Graduating in December 2023 from the \textbf{University of Toronto} with a major in \textbf{Math and Physics} and a minor in \textbf{Computer Science}, I am eager to utilize what I have learned and showcase my aptitude for further learning to you.

I am currently working as a Software Engineer Intern at \textbf{Capital One}, where I am part of the Payments Intelligence team. In this role, I am working on the migration of our cache from Redis to DynamoDB to facilitate in \textbf{annual savings of over \$100,000}. I am working with a \textbf{gRPC API} that interfaces with our machine learning model while leveraging various \textbf{AWS} tools such as DynamoDB, EC2, ECS, Fargate, S3, Boto3, and ElastiCache. For an application handling \textbf{over 180 billion transactions each year}, this experience is allowing me to deepen my knowledge of scalable systems and hone my skills in utilizing cutting-edge technologies.

During my last internship at Promise Robotics, a \textbf{seed-stage startup}, I gained invaluable exposure to cutting-edge technology and the excitement of building upon proof of concept. Working on automating construction using robots, I collaborated on preprocessing CAD models, designed algorithms for robot sequencing, and implemented machine learning techniques to optimize the process. Additionally, I played an integral role in enhancing the security of the API. Most notably, I was able to apply Quaternion mathematics to sequence irregular shapes in roofs and windows.

I am also a \textbf{ML Research Assistant} at the Cognitive Neuroscience \& Sensorimotor Integration (CoNSens) Laboratory at the University of Toronto. I am utilizing ML to understand the neuroscience of the Dorsal and Ventral Streams. With a focus on human grasp point determination and object recognition, my work involves exploring \textbf{explainable convolutional neural networks (CNNs)} and analyzing EEG data. Through this research, I am investigating the optimization strategies employed by these two streams, aiming to understand their differences beyond initializations. This experience allows me to use my previous University studies in \textbf{Biology \& Chemistry} with my current aspirations in tech that fits my holistic way of thinking. 

During the school year, I was given a unique experience in technology as an \textbf{IT Support Assistant} for three biological departments at the University of Toronto. In a division of 4 people, I was given a large responsibility in maintaining the integrity and efficiency of our critical systems as well as creating new embedded systems for the departments. 

Aside from professional experience, I am a \textbf{Senior Strategy Engineer at Blue Sky Solar Racing} where I optimized the construction, telemetry, and performance of our solar car for international competitions. As a member of the UofT Aerospace Team's Space System Division's Optics Team, I perform numerical analysis and conduct research on grisms and holographic gratings for our satellite payload. The payload will be integrated into a \textbf{hyperspectral imaging CubeSat} to measure anthropogenic gas emissions across Ontario in 2025. Moreover, I have developed several projects, including a \textbf{webcam diagnostic tool for diagnosing strabismus} in collaboration with researchers at the University of Calgary and a \textbf{friend-making web app called Amigos} deployed on Azure, showcasing my technical abilities in computer vision and web development.
% Furthermore, as a Senior Strategy Engineer at Blue Sky Solar Racing, I optimized the construction, telemetry, and performance of our solar car for prestigious competitions. This involved conducting research on the implementation of bifacial solar cells and creating simulations to analyze cell output based on weather and geographic data. Additionally, I developed a parallel computed simulation using MATLAB, Python, Ansys, and CAD tools, which significantly contributed to the team's strategic planning. As an active member of the UofT Aerospace Team's Space System Division, I contribute to the optics team, performing numerical analysis using Python and conducting research on grisms and holographic gratings for our satellite payload. Currently, I am leading a team of two members in designing tests for an optical bench prototype, which will be integrated into a hyperspectral imaging CubeSat to measure anthropogenic gas emissions across Ontario, Canada in 2024. Moreover, I have developed several impressive projects, including a webcam diagnostic tool for diagnosing strabismus in collaboration with researchers at the University of Calgary and a friend-making web app called Amigos deployed on Azure, showcasing my technical abilities in computer vision and web development.
% As a Senior Strategy Engineer at Blue Sky Solar Racing, I undertook comprehensive optimization efforts to enhance the construction, telemetry, and performance of our solar car for the American and World Solar Competitions. I conducted research on the implementation of bifacial solar cells and leveraged simulations to predict cell output based on weather and geographic data. Additionally, I created a parallel computed simulation using MATLAB, Python, Ansys, and CAD, empowering future generations to refine their strategies.

% Beyond my professional experiences, I have engaged in several other exciting projects. For instance, I collaborated with Dr. Etienne Benard-Seguin and Jeremy Moreau from the University of Calgary to develop a webcam diagnostic tool for diagnosing strabismus using OpenCV and mediapipe. I also developed Amigos, a friend-making web app, using Java, Spring Framework, Figma, Thymeleaf, JavaScript, CSS, and HTML, which was deployed on Azure. These projects have showcased my adaptability, creativity, and commitment to leveraging technology for positive impact.

% In addition to my technical prowess, I am a well-rounded individual with a passion for continuous learning and personal growth. I am fluent in four languages, play three instruments, and have competed in varsity sports such as wrestling and track. These experiences have fostered my ability to work effectively in diverse teams, adapt to new challenges, and think critically.

% I am confident that my strong technical skills, research experience, and multifaceted background make me a strong candidate for a software engineering role at your organization. I am eager to contribute my knowledge, creativity, and dedication to achieving outstanding results and driving innovation.

Additionally, I am also furthering my learning in \textbf{Finance} including being \textbf{Bloomberg Market Concepts} certified and self-learning in my free time. I am a light-hearted ENTP character with a unique perspective. If there is ever something that I do not know, I \textbf{guarantee} that I can \textbf{learn it and demonstrate that to you}. Thank you for considering my application. I would welcome the opportunity to discuss how my qualifications align with your team's needs in greater detail.






%   Currently, I am working as an information technology assistant for 
%    the IT department of 3 biological departments at UofT. I am working in \textbf{Linux} to
%    maintain the robustness and security of our servers and computer architecture. In the summer of 2022, I had the opportunity to work as
%     a Software Engineer at \textbf{Promise Robotics} for 4 months in 
%     utilizing robotics for automated construction. 
%     I utilized \textbf{Python, Django, React, Docker, Machine 
%     Learning, and applied Physics} as the top intern 
%     contributor.  I had the opportunity to work
%         with programming the robots directly as well as full 
%         stack development and code reviews with the databases and security of
%          our API. Besides that, I am a senior strategy 
%          engineer for the Blue Sky Solar Racing Team at 
%          the University of Toronto. I conduct solar cell
%           research (optics and bifacial cells), fabricate,
%            and create simulations \textbf{(Matlab, Python, Parallel 
%            Computing \& CAD)} to optimize the performance and
%             design of a solar car. I am also on the optics 
%             team of the UofT Aerospace Team Space System
%              Division where I do numerical analysis \textbf{(Python)} 
%              and optical research (grisms and holographic 
%              gratings) of our satellite payload. I am also leading a team of 2 to 
%              design tests for our optical bench prototype
%               which is going to be part of a hyperspectral 
%               imaging CubeSat to measure anthropogenic gas 
%               emissions across Ontario, Canada in 2024.
%                Some projects of mine include a webcam 
%                diagnostic tool in coordination with Dr.
%                 Etienne Benard-Seguin and Jeremy Moreau
%                  (University of Calgary) to diagnose 
%                  strabismus using OpenCV and mediapipe. 
%                  Another project of mine that I developed
%                   was a friend-making webapp called Amigos
%                    using \textbf{Java, Spring Framework, Figma, Thymeleaf, Javascript, CSS, and HTML} and deployed on \textbf{Azure}.
%                     I have some skills but more importantly,
%                      have an aptitude for learning that I 
%                      would like to demonstrate to you. I am
%                       a well-rounded person that speaks 4
%                       languages, plays 3 instruments, and 
%                       played 2 varsity sports (wrestling and 
%                       track). \par
%      Overall, I would like to state that:


% -I have the \bold{Skills:} I am a critical thinker and eager learner with mastery in
% \bold{Python, Java, and MATLAB}. I am also currently working towards mastering
% \bold{C and C++} and am working towards a Self Driving Car 4-Course Specialization
% through the University of Toronto.\par

% -I have the \bold{Passion:} I spend the bulk of my free time curiously learning 
% on the internet and talking to friends in the field. All part of the fun!\par
% -I have the \bold{Character:} I am a lighthearted guy. I find pleasure in the
%  connections and conversations in life  :)




%%%%%%% --------------------------------------------------------------------------------------
%%%%%%%  SIGNATURE
%%%%%%% --------------------------------------------------------------------------------------

\vspace{0.5cm}
\raggedright
Sincerely, \\ Lawrence (Larry) Tsai %\\ (267)-401-2843 \\ \href{mailto:L.tsai@mail.utoronto.ca}{L.tsai@mail.utoronto.ca}

\end{document}